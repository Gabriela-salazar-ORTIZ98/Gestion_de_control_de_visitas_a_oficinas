\documentclass[10pt,a4paper]{article}
\usepackage[utf8]{inputenc}
\usepackage{amsmath}
\usepackage{amsfonts}
\usepackage{amssymb}
\begin{document}
\section{INTRODUCTION!}
La gestión del control de visitas a oficinas es una parte integral de la gestión y seguridad de espacios corporativos y comerciales. En un mundo en constante evolución donde la tecnología y la eficiencia son claves, la capacidad de administrar y monitorear las visitas de manera efectiva se ha convertido en una prioridad para muchas empresas. Desde pequeñas empresas hasta grandes corporaciones, la implantación de un sistema de gestión de control de visitas no solo aumenta la seguridad, sino que también optimiza los procesos internos y fortalece la imagen de la empresa.
El control de asistencia de la oficina cubre muchos aspectos, desde recibir invitados e identificar proveedores hasta monitorear reuniones y administrar el acceso a áreas restringidas. Este control se logra a través de tecnologías y herramientas innovadoras que combinan software y hardware para crear un sistema eficiente y seguro. Los sistemas tradicionales de registro de visitantes en papel han evolucionado hacia soluciones digitales más avanzadas que ofrecen mayor comodidad y funcionalidad.

\section{Objective general!}
Desarrollar e implementar un sistema de gestión de control de visitas a oficinas para mejorar la seguridad, eficiencia y organización en el registro y seguimiento de visitantes, agilizar procesos internos y fortalecer la imagen de la organización.

\section{Objectives específicos!}
*Diseñar una interfaz de sistema de administración de control de visitas intuitiva y fácil de usar que permita a los usuarios registrar y administrar la información de los visitantes de manera eficiente.

*Desarrollar un sistema de registro de visitantes en línea que permita a los invitados registrarse previamente antes de su llegada, simplificando el proceso de registro en el sitio.

*Configurar un sistema de notificaciones automáticas que alerte a los anfitriones sobre la llegada de sus visitantes y les proporcione detalles relevantes para facilitar la recepción.

*Configurar los niveles de acceso y los permisos para los usuarios del sistema y asegúrese de que solo las personas autorizadas puedan acceder a la información de los visitantes.

*Capacitar el personal encargado de recibir y gestionar las visitas sobre el uso adecuado del sistema para asegurar su correcta implementación y operación continua.
\end{document}